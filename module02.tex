\documentclass[ignorenonframetext,]{beamer}
\setbeamertemplate{caption}[numbered]
\setbeamertemplate{caption label separator}{: }
\setbeamercolor{caption name}{fg=normal text.fg}
\beamertemplatenavigationsymbolsempty
\usepackage{lmodern}
\usepackage{amssymb,amsmath}
\usepackage{ifxetex,ifluatex}
\usepackage{fixltx2e} % provides \textsubscript
\ifnum 0\ifxetex 1\fi\ifluatex 1\fi=0 % if pdftex
  \usepackage[T1]{fontenc}
  \usepackage[utf8]{inputenc}
\else % if luatex or xelatex
  \ifxetex
    \usepackage{mathspec}
  \else
    \usepackage{fontspec}
  \fi
  \defaultfontfeatures{Ligatures=TeX,Scale=MatchLowercase}
\fi
% use upquote if available, for straight quotes in verbatim environments
\IfFileExists{upquote.sty}{\usepackage{upquote}}{}
% use microtype if available
\IfFileExists{microtype.sty}{%
\usepackage{microtype}
\UseMicrotypeSet[protrusion]{basicmath} % disable protrusion for tt fonts
}{}
\newif\ifbibliography
\hypersetup{
            pdftitle={Module 02 - The Input Output Model},
            pdfauthor={Renato Vargas},
            pdfborder={0 0 0},
            breaklinks=true}
\urlstyle{same}  % don't use monospace font for urls

% Prevent slide breaks in the middle of a paragraph:
\widowpenalties 1 10000
\raggedbottom

\AtBeginPart{
  \let\insertpartnumber\relax
  \let\partname\relax
  \frame{\partpage}
}
\AtBeginSection{
  \ifbibliography
  \else
    \let\insertsectionnumber\relax
    \let\sectionname\relax
    \frame{\sectionpage}
  \fi
}
\AtBeginSubsection{
  \let\insertsubsectionnumber\relax
  \let\subsectionname\relax
  \frame{\subsectionpage}
}

\setlength{\parindent}{0pt}
\setlength{\parskip}{6pt plus 2pt minus 1pt}
\setlength{\emergencystretch}{3em}  % prevent overfull lines
\providecommand{\tightlist}{%
  \setlength{\itemsep}{0pt}\setlength{\parskip}{0pt}}
\setcounter{secnumdepth}{0}

\title{Module 02 - The Input Output Model}
\author{Renato Vargas}
\date{6/20/2018}

\begin{document}
\frame{\titlepage}

\begin{frame}{Fundamental relationships}

We categorize the economy in \emph{n} sectors. \(x_i\) is the total
output of sector \(i\) and \(f_i\) is the total demand for sector
\(i\)'s product. So sales of this sector to other sectors and final
demand for all n sectors can be expressed as:

\[
\begin{equation}
x_j=z_{11}+\cdots+z_{1j}+\cdots+z_{1n}+f_i
\end{equation}
\] \[
\begin{equation}
x_j=z_{i1}+\cdots+z_{ij}+\cdots+z_{in}+f_i
\end{equation}
\] \[
\begin{equation}
x_j=z_{n1}+\cdots+z_{nj}+\cdots+z_{nn}+f_i
\end{equation} 
\]

\end{frame}

\begin{frame}{Compact representation}

\[\begin{equation} \mathbf{x} = \begin{bmatrix} \begin{array}{c} x_1 \\ \vdots\\ x_n\\ \end{array} \end{bmatrix} \mathbf{Z} = \begin{bmatrix}\begin{array}{ccc} z_{11} & \ldots & z_{1n} \\ \vdots &  \ddots &  \\ z_{n1} &  & z_{nn} \end{array} \end{bmatrix} \mathbf{f}= \begin{bmatrix} \begin{array}{c} f_{1} \\ \vdots  \\ f_{n} \end{array} \end{bmatrix}\end{equation}\]

or

\(\mathbf{x} = \mathbf{Zi} + \mathbf{f}\)

 We need to multiply by the summation column vector \(\mathbf{i}\) of
1's of dimension n.

\end{frame}

\begin{frame}{Input-output coefficient}

Given \(z_{ij}\) and \(x_j\), for example input of boat fuel (\(i\))
bought by small-scale fishers (\(j\)) last year and total fishers
production last year - form the ratio of fuel of fuel to fishers output,
\(z_{ij}/x_{j}\), (both units in currency) and denote it \(a_{ij}\).

Consequently, direct input (technical) coefficients can be defined as

\[a_{ij} = \frac{z_{ij}}{x_j}\]

This ratio is called a technical coefficient, input-output coefficient,
or direct input coefficient.

\end{frame}

\begin{frame}{Coefficients for all sectors}

We have that \(\mathbf{x} = \mathbf{Zi} + \mathbf{f}\).

We know that a hat over a vector \(\mathbf{\hat{x}}\) yields a matrix
with the vector along its diagonal.

We know that \((\mathbf{\hat{x}})(\mathbf{\hat{x}})^{-1} = \mathbf{I}\)
so it follows that \(\mathbf{\hat{x}}^-1\)

Therefore the n x n matrix of technical coefficients can be represented
as

\[\mathbf{A} = \mathbf{Z}\mathbf{\hat{x}}^{-1}\]

\end{frame}

\begin{frame}{Example}

\end{frame}

\end{document}
